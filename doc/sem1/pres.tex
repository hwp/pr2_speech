\documentclass{beamer}

\title{Sound Source Localization and Separation}
\author{Weipeng He}

\usetheme{Warsaw}

\begin{document}

\frame{\titlepage}

\begin{frame}
  \frametitle{Task}
  \begin{block}{Input}
  Multichannel audio signal (from KINECT);
  \end{block}
  ~
  \begin{block}{Outputs}
  The direction of arrival and separated signal of each sound source; 
  \end{block}
  ~
  \begin{block}{Meaning}
  Reducing the noise in the environment; \\
  Better signal for speech recognition.
  \end{block}
\end{frame}

\begin{frame}
  \frametitle{Result of Last Semester}
  
  \begin{itemize}
    \item Sound source localization using HARK.
      \begin{itemize}
        \item No significant correct result:
        \item Noise from the environment;
        \item Distance between the microphone is short.
      \end{itemize}
    ~

    \item Sound source separation using HARK.
      \begin{itemize}
        \item Depends on the result of localization.
      \end{itemize}
  \end{itemize}
\end{frame}

\begin{frame}
  \frametitle{Limitations and Improvement}
  
  \begin{itemize}
     \item Flow design in HARK, hard to customize;
     \item Impossible to change algorithm;

     ~
     
     \item Use the HARK-Kinect driver.
     \item Implement an algorithm\footnote{Otsuka, T., Ishiguro, K., Sawada, H., & Okuno, H. G. (2012). Bayesian Unification of Sound Source Localization and Separation with Permutation Resolution. In \textit{Proc. of AAAI Conf. on Artificial Intelligence} (pp. 2038-2045).} by hand.
     \item (OPTIONAL) Use customized microphone array.
  \end{itemize}
  
\end{frame}

\end{document}
